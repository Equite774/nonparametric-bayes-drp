\documentclass{chaistyle}

\setcoursename{DRP Spring 2025}
\settitle{Nonparametric Bayes}
\setname{Chai Harsha}

\title{Nonparametric Bayes DRP Notes}

\begin{document}
\maketitle
\section*{Day 1 - 2/6/25}
\begin{definition*}
    Let \(X_1, X_2, \dots, X_n\) be a sequence of independent and identically distributed (i.i.d.) random variables with common CDF \(F\). The \tbf{empirical distribution} function is defined as \[\hat F^n(x)=\frac{1}{n} \sum_{i=1}^n\indic_{X_i\leq x}.\]
\end{definition*}
\begin{theorem*}[Law of Large Numbers]
    \[\lim_{n \to \infty} \hat F^n(x)\to F(x)\] with probability 1.
\end{theorem*}
\begin{theorem*}[Central Limit Theorem]
    \[\lim_{n \to \infty} \sqrt n \hat F^n(x)\to N(F(x),F(x)(1-F(x)))\] with probability 1.
\end{theorem*}
\begin{theorem*}[Glivenko-Cantelli Theorem]
    \[\lim_{n \to \infty} \hat F^n\to F\] uniformly with probability 1. \[P(\sup_x|\hat F^n(x)-F(x)|\to 0)=1.\]
\end{theorem*}
\subsection*{What is Probability?}
Bayesian probability is a measure of the plausibility of an event given incomplete knowledge. Frequentist probability is a measure of the frequency of an event in a large number of trials. Both approaches can be applied to statistics.
\subsection*{Statistics}
One truth \(\mu,\) along with random data.
\begin{itemize}[-]
    \item Frequentists exclusively base their conclusions on repeated sampling.
    \item What if you can't smaple the data repeatedly? What is the probability that a team wins the Super Bowl in a given year?
    \item Bayesian argument - the level of belief in an event.
    \item In statistics, we have our observations \(X_1, X_2, \dots, X_n\) which are fixed, and we repeatedly update \(\mu.\)
    \item To summarize, frequentists view the data is random and the truth is fixed, Bayesians fix the data while the truth is random.
\end{itemize}
Our framework is as follows:
\[\{X_i\}_{i=1}^n\sim p(\theta)=p(x|\theta)\] where our prior distribution is \(p(\theta)\) and our likelihood function is \(p(x|\theta).\) The posterior distribution is \[p(\theta|x)\propto \frac{p(x|\theta)p(\theta)}{\int p(x|\theta)p(\theta)\,d\theta}.\]
\
If \(\theta\) is a function, what is \(p(\theta)\)? If you can compute it, how do you compute \(\int p(x|\theta)p(\theta)\,d\theta\)?
\subsection*{Overview}
\begin{itemize}[-]
    \item \textbf{Theory}
    \begin{enumerate}
        \item Exchangeability - Our data is drawn from a conditional distribution, so we are really assuming that it is conditionally independent. \(\{X_i\}_{i=1}^n\) are technically dependent! Di Finetti Theorem - Conditionally iid \(\iff\) exchangeability.
        \item Frequentist guarantees - If we take the limit \(n\to\infty,\) we want to approach the truth. We can't know everything, so we need to know how close we are to the truth, even if the proof of this is finnicky.
    \end{enumerate}
    \item \textbf{Computation}
    \begin{enumerate}
        \item Conjugacy - We can get around the integral \(\int p(x|\theta)p(\theta)\,d\theta\) by choosing a prior that is conjugate to the likelihood function, which will save us from having to compute the integral analytically.
        \item MCMC - Markov Chain Monte Carlo - We can sample from the posterior distribution using MCMC methods.
    \end{enumerate}
\end{itemize}
\section*{Day 2 - 2/13/25}
We study single-parameter models. There are four models which we will consider: binomial, normal, Poisson, and exponential.
\begin{enumerate}[1.]
    \item \textbf{Binomial}
    
    We aim to estimate the population proportion from a sequence of Bernoulli trials (each data \(y_1,\dots,y_n\in\{0,1\}\)). Order does not matter (i.e. the data is \textbf{exchangeable}), so the model is defined by \[p(y|\theta)=\text{Bin}(y|n,\theta)=\binom{n}{y}\theta^y(1-\theta)^{n-y}\] where \(\theta\) is the probability of success, \(n\) is the number of trials, and \(y\) is the number of successes (\(y\leq n\)).
    \vspace{3mm}
    \begin{example*}[Probability of Female Birth]

        We define \(\theta\) to be the proportion of female births. Hence, \(1-\theta\) is the proportion of male births.Let \(y\) be the number of female births among \(n\) recorded births.

        We need a prior distribution for \(\theta.\) For our purposes, \(p(\theta)\sim\text{Unif}([0,1]).\)

        From this, through Bayes' Law and removing constant terms w.r.t. the parameter, we obtain the posterior distribution \[p(\theta|y)\propto\theta^y(1-\theta)^{n-y}.\] However, in the case of a binomial distribution with uniform prior, we may explicitly calculate \(p(y).\)

        Once we have calculated the posterior, in order to make predictions under the above conditions, we have \begin{align*}
            \P(y_{n+1}=1|y) &= \int_{0}^{1}\P(y_{n+1}=1|\theta,y)p(\theta|y)\,d\theta \\ 
            &= \int_{0}^{1}\theta\cdot p(\theta|y)\,d\theta \\ 
            &= \E(\theta|y) \\ 
        \end{align*}
        The posterior incorporates information from the data, so it will be less variable than the prior. We formalize as the Tower Property: \[\E(\theta)=\E(\E(\theta|y))\] and \[\text{Var}(\theta)=\E(\text{Var}(\theta|y)) + \text{Var}(\E(\theta|y)).\]

        How might we interpret the prior distribution? How might we select it? 
        \begin{itemize}[-]
            \item Population interpretation - the prior is a population of possible parameter values, from which the current was selected.
            \item State of knowledge interpretation - the prior distribution represents our knowledge about the parameter. A greater variance means that we know more about the underlying distribution.
        \end{itemize}
        A prior distribution that is of the same form as the posterior is called \tbf{conjugate}.
    \end{example*}
\end{enumerate}
\section*{Day 3 - 2/20/25}
The key will always be \[p(\theta|y)=\frac{p(y|\theta)p(\theta)}{p(y)}.\] For today, we will be using \[p(h|D)=\frac{p(D|h)p(h)}{p(D)}.\] The prior represents our information about our posterior. In many cases, the prior is uniform, which means \[p(h|D)\propto p(D|h).\] 

The \textit{probability} of an event is \[\int_{x-\delta}^{x+\delta}f(y)\,dy,\] while the \textit{likelihood} is just \(f(x).\)

The \textit{maximum likelihood estimator} (LME) is, given \((X_n)_{i=1}^N\sim p(\cdot|\theta),\) is the parameter \begin{align*}
    \hat\theta&=\text{argmax}_\theta\prod_{i=1}^Np(X_i|\theta) \\ 
    &= \text{argmax}_\theta \lim_{\delta \to 0} \prod_{i=1}^N \frac{\P_\theta(Y\in X_i\pm\delta)}{\delta}
\end{align*}
where \(Y\sim p(\cdot,\theta).\)
The steps are: \begin{enumerate}
    \item[-1.] Determine the model \(p(\cdot,\theta).\) We are picking a class of function, for which \(\theta\) is a parameter.
    \item[0.] Generate \((X_i)_{i=1}^N\) from our model.
    \item[1.] For each \(\theta\in\R,\) compute the likelihood of seeing \((X_i)_{i=1}^N\) using \[\mathcal{L}(X,\theta)=\prod_{i=1}^N p(X_i|\theta).\]
    \item[2.] Choose the \(\theta\) that maximizes \(\mathcal{L}.\)
\end{enumerate}
The \textit{maximal a posteriori} (MAP) estimator is the same, but we maximize the posterior distribution instead of the likelihood function: \[\hat\theta_{MAP}=\text{argmax}_\theta \prod_{i=1}^N p(\theta|D).\]
As we increase the amount of data we have to \(\infty\), the MAP estimator converges to the MLE. Intuitively, the posterior is proportional to the likelihood, and with more data, the likelihood term dominates the prior.

\begin{example*}
    Let \(N_1\) be the number of heads, and \(N\) be the total number of tosses. Let \(a,b\) be hyperparameters. Then \begin{align*}
        \hat\theta_{MAP} &= \text{argmax}_{\theta\in[0,1]}\frac{\theta^{N_1+a-1}(1-\theta)^{N-N_1+b-1}}{p(D)} \\ 
        &= \text{argmax}_{\theta\in[0,1]}(N_1+a-1)\log(\theta)+(N-N_1+b-1)\log(1-\theta) \\ 
    \end{align*}
    We set \[\frac{\del g(\theta)}{\del \theta}=0.\]
    \begin{align*}
        0 &= \frac{N_1+a-1}{\theta}-\frac{N-N_1+b-1}{1-\theta} \\
        &= (1-\theta)(N_1+a-1)-\theta(N-N_1+b-1) \\
        \theta(N+a+b-2) &= N_1+a-1 \\ 
        \hat\theta_{MAP} &= \frac{N_1+a-1}{N+a+b-2}.
    \end{align*}
\end{example*}
\end{document}